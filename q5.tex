\section*{Quiz 5 (Feb 21, 2020)}

\problem{
    w20q5q1
    \xxxxxx
}{
    Assuming a 5-way multiple issue processor with a 9-stage pipeline and no bubbles, how many instructions are in execution at any time?
}{
    $ 5 \times 9 = 45 $
}

\problem{
    w20q5q2
    \xxxxxx
}{
    As of 2011, how many processors combine full speculation with resolving multiple branches per cycle?
}{
    No processor
}

\problem{
    w20q5q3
    \xxxxxx
}{
    Tomasulo's algorithm attempts to achieve something called Data Flow execution. 
    How does the book define Data Flow execution?
}{
    \pg{184} Operation executes as soon as their operands are available.
}

\problem{
    w20q5q4
    \xxxxxx
}{
    We have talked about two types of scheduling(static and dynamic). 
    Which type does a superscalar machine use? 
    Which one is primarily used in a VLIW machine?
}{
    VILW: static, \\
    Superscalar: both static and dynamic,
}

\problem{
    w20q5q5
    \xxxxxx
}{
    The book talks about the limits of ILP and mentions that even when using a perfect model there are some limitations. 
    What are the most important limitations that apply even to the perfect model?
}{
    \pg{220}
    \begin{enumerate}
        \item WAW \& WAR through the memory system
        \item Unnecessary dependencies
        \item Overcoming data flow limits
    \end{enumerate}
}

\problem{
    w20q5q6
    \xxxxxx
}{
    Processors have been built that are able to issue 8 instructions at a time. 
    However, these pro­cessors are no longer being built - why not? 
    Why would you choose a 3-issue machine over an 8-issue machine?
}{
    \todo
}

\problem{
    w20q5q7
    \xxxxxx
}{
    Register renaming is used to avoid name hazards. 
    Is there a technique that can be used to minimize the number of stalls due to true (RAW) hazards? 
    If so, what is it and how does it work?
}{
    \todo
}
