\section*{Week 8 (2/24 - 3/1; Chp.4 \& Appx.B, Midterm 2)}

\problem{
}{
    What is a chime?
}{
    \pg{269} The unit of time taken to execute one convoy.
}

\problem{
}{
    What is the critical advantage of a vector instruction set?
}{
    It allows the software to pass a large amount of parallel work to hardware using only a single short instruction.
}

\problem{
    f15m2q11
    f15f0p4
    f16m2q10
    w19m2q2
    w19f0q33
}{
    Briefly outline how a Vector machine works, and what type of parallelism it is exploiting.
}{
    There are several vector registers(VR), fill them up. 
    Single instruction operated on these VRs, such that the single instruction do multiple operations on elements of the VRs. 

    Data level parallelism.
}

\problem{
    w13f0p3
    w15f0p3
}{
    Does the design of the instruction set have a significant impact on the ability to pipeline a processor?
    If so, explain your answer \& give an example.
}{
    Yes, \todo

    Vector machines
}

\problem{
    w19m2q7
    w20m2q7
 }{
    A vector processor is a type of SIMD architecture. 
    What makes it different than a ``normal'' SIMD machine? 
    (Why does it have a separate section in the SIMD chapter in the book?)
}{
    Vector computers processed the vectors one word at a time through pipelined processors 
    (though still based on a single instruction)
    Whereas modern SIMD computers process all elements of the vector simultaneously.
}

\problem{
}{
    Is it true that you can get a good vector performance without providing memory bandwidth? 
    Why or why not?
}{
    No.

    Memory bandwidth is important to all SIMD architectures due to large amount of memory references.
}

\problem{
}{
    Loops can be vectorized when they don't have what?
}{
    Dependences between iterations of a loop, or loop-carried dependences.
}

\problem{
    w15f0p2
}{
    The NVIDIA GPU exploits multiple \blank to achieve maximum performance
}{
    thread SIMD
}